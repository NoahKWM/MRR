% Options for packages loaded elsewhere
\PassOptionsToPackage{unicode}{hyperref}
\PassOptionsToPackage{hyphens}{url}
%
\documentclass[
]{article}
\usepackage{amsmath,amssymb}
\usepackage{iftex}
\ifPDFTeX
  \usepackage[T1]{fontenc}
  \usepackage[utf8]{inputenc}
  \usepackage{textcomp} % provide euro and other symbols
\else % if luatex or xetex
  \usepackage{unicode-math} % this also loads fontspec
  \defaultfontfeatures{Scale=MatchLowercase}
  \defaultfontfeatures[\rmfamily]{Ligatures=TeX,Scale=1}
\fi
\usepackage{lmodern}
\ifPDFTeX\else
  % xetex/luatex font selection
\fi
% Use upquote if available, for straight quotes in verbatim environments
\IfFileExists{upquote.sty}{\usepackage{upquote}}{}
\IfFileExists{microtype.sty}{% use microtype if available
  \usepackage[]{microtype}
  \UseMicrotypeSet[protrusion]{basicmath} % disable protrusion for tt fonts
}{}
\makeatletter
\@ifundefined{KOMAClassName}{% if non-KOMA class
  \IfFileExists{parskip.sty}{%
    \usepackage{parskip}
  }{% else
    \setlength{\parindent}{0pt}
    \setlength{\parskip}{6pt plus 2pt minus 1pt}}
}{% if KOMA class
  \KOMAoptions{parskip=half}}
\makeatother
\usepackage{xcolor}
\usepackage[margin=1in]{geometry}
\usepackage{color}
\usepackage{fancyvrb}
\newcommand{\VerbBar}{|}
\newcommand{\VERB}{\Verb[commandchars=\\\{\}]}
\DefineVerbatimEnvironment{Highlighting}{Verbatim}{commandchars=\\\{\}}
% Add ',fontsize=\small' for more characters per line
\usepackage{framed}
\definecolor{shadecolor}{RGB}{248,248,248}
\newenvironment{Shaded}{\begin{snugshade}}{\end{snugshade}}
\newcommand{\AlertTok}[1]{\textcolor[rgb]{0.94,0.16,0.16}{#1}}
\newcommand{\AnnotationTok}[1]{\textcolor[rgb]{0.56,0.35,0.01}{\textbf{\textit{#1}}}}
\newcommand{\AttributeTok}[1]{\textcolor[rgb]{0.13,0.29,0.53}{#1}}
\newcommand{\BaseNTok}[1]{\textcolor[rgb]{0.00,0.00,0.81}{#1}}
\newcommand{\BuiltInTok}[1]{#1}
\newcommand{\CharTok}[1]{\textcolor[rgb]{0.31,0.60,0.02}{#1}}
\newcommand{\CommentTok}[1]{\textcolor[rgb]{0.56,0.35,0.01}{\textit{#1}}}
\newcommand{\CommentVarTok}[1]{\textcolor[rgb]{0.56,0.35,0.01}{\textbf{\textit{#1}}}}
\newcommand{\ConstantTok}[1]{\textcolor[rgb]{0.56,0.35,0.01}{#1}}
\newcommand{\ControlFlowTok}[1]{\textcolor[rgb]{0.13,0.29,0.53}{\textbf{#1}}}
\newcommand{\DataTypeTok}[1]{\textcolor[rgb]{0.13,0.29,0.53}{#1}}
\newcommand{\DecValTok}[1]{\textcolor[rgb]{0.00,0.00,0.81}{#1}}
\newcommand{\DocumentationTok}[1]{\textcolor[rgb]{0.56,0.35,0.01}{\textbf{\textit{#1}}}}
\newcommand{\ErrorTok}[1]{\textcolor[rgb]{0.64,0.00,0.00}{\textbf{#1}}}
\newcommand{\ExtensionTok}[1]{#1}
\newcommand{\FloatTok}[1]{\textcolor[rgb]{0.00,0.00,0.81}{#1}}
\newcommand{\FunctionTok}[1]{\textcolor[rgb]{0.13,0.29,0.53}{\textbf{#1}}}
\newcommand{\ImportTok}[1]{#1}
\newcommand{\InformationTok}[1]{\textcolor[rgb]{0.56,0.35,0.01}{\textbf{\textit{#1}}}}
\newcommand{\KeywordTok}[1]{\textcolor[rgb]{0.13,0.29,0.53}{\textbf{#1}}}
\newcommand{\NormalTok}[1]{#1}
\newcommand{\OperatorTok}[1]{\textcolor[rgb]{0.81,0.36,0.00}{\textbf{#1}}}
\newcommand{\OtherTok}[1]{\textcolor[rgb]{0.56,0.35,0.01}{#1}}
\newcommand{\PreprocessorTok}[1]{\textcolor[rgb]{0.56,0.35,0.01}{\textit{#1}}}
\newcommand{\RegionMarkerTok}[1]{#1}
\newcommand{\SpecialCharTok}[1]{\textcolor[rgb]{0.81,0.36,0.00}{\textbf{#1}}}
\newcommand{\SpecialStringTok}[1]{\textcolor[rgb]{0.31,0.60,0.02}{#1}}
\newcommand{\StringTok}[1]{\textcolor[rgb]{0.31,0.60,0.02}{#1}}
\newcommand{\VariableTok}[1]{\textcolor[rgb]{0.00,0.00,0.00}{#1}}
\newcommand{\VerbatimStringTok}[1]{\textcolor[rgb]{0.31,0.60,0.02}{#1}}
\newcommand{\WarningTok}[1]{\textcolor[rgb]{0.56,0.35,0.01}{\textbf{\textit{#1}}}}
\usepackage{graphicx}
\makeatletter
\def\maxwidth{\ifdim\Gin@nat@width>\linewidth\linewidth\else\Gin@nat@width\fi}
\def\maxheight{\ifdim\Gin@nat@height>\textheight\textheight\else\Gin@nat@height\fi}
\makeatother
% Scale images if necessary, so that they will not overflow the page
% margins by default, and it is still possible to overwrite the defaults
% using explicit options in \includegraphics[width, height, ...]{}
\setkeys{Gin}{width=\maxwidth,height=\maxheight,keepaspectratio}
% Set default figure placement to htbp
\makeatletter
\def\fps@figure{htbp}
\makeatother
\setlength{\emergencystretch}{3em} % prevent overfull lines
\providecommand{\tightlist}{%
  \setlength{\itemsep}{0pt}\setlength{\parskip}{0pt}}
\setcounter{secnumdepth}{-\maxdimen} % remove section numbering
\ifLuaTeX
  \usepackage{selnolig}  % disable illegal ligatures
\fi
\IfFileExists{bookmark.sty}{\usepackage{bookmark}}{\usepackage{hyperref}}
\IfFileExists{xurl.sty}{\usepackage{xurl}}{} % add URL line breaks if available
\urlstyle{same}
\hypersetup{
  pdftitle={TP2 MRR},
  pdfauthor={Noah KWA MOUTOME - Victor TAN},
  hidelinks,
  pdfcreator={LaTeX via pandoc}}

\title{TP2 MRR}
\author{Noah KWA MOUTOME - Victor TAN}
\date{2023-10-31}

\begin{document}
\maketitle

\section{IV. Cookies Study}\label{iv.-cookies-study}

First, let's go back to our previous results: We saw that with linear
regression without penalty, only 2 explanatory variables were
significant. We can deduce that we can also use a Lasso regression to
select the most important features to predict the fat value of a cookie.
Futhermore, we can use a Ridge regression to see if the model is
overfitting or not.

\subsection{Imports}\label{imports}

\begin{Shaded}
\begin{Highlighting}[]
\NormalTok{cookies\_data }\OtherTok{\textless{}{-}} \FunctionTok{read.csv}\NormalTok{(}\StringTok{"cookies.csv"}\NormalTok{)}
\end{Highlighting}
\end{Shaded}

\subsection{Features extraction}\label{features-extraction}

For each line (meaning, for each cookie), we will use the different
spectral values to compute: the mean, the standard deviation, the slope,
the minimum and the maximum.

\begin{Shaded}
\begin{Highlighting}[]
\CommentTok{\# Computation (mean, standard deviation, minimum and maximum)}

\NormalTok{cookies\_data}\SpecialCharTok{$}\NormalTok{mean }\OtherTok{\textless{}{-}} \FunctionTok{rowMeans}\NormalTok{(cookies\_data[, }\DecValTok{2}\SpecialCharTok{:}\DecValTok{701}\NormalTok{])}
\NormalTok{cookies\_data}\SpecialCharTok{$}\NormalTok{stDev }\OtherTok{\textless{}{-}} \FunctionTok{apply}\NormalTok{(cookies\_data[, }\DecValTok{2}\SpecialCharTok{:}\DecValTok{701}\NormalTok{], }\DecValTok{1}\NormalTok{, sd)}
\NormalTok{cookies\_data}\SpecialCharTok{$}\NormalTok{min }\OtherTok{\textless{}{-}} \FunctionTok{apply}\NormalTok{(cookies\_data[, }\DecValTok{2}\SpecialCharTok{:}\DecValTok{701}\NormalTok{], }\DecValTok{1}\NormalTok{, min)}
\NormalTok{cookies\_data}\SpecialCharTok{$}\NormalTok{max }\OtherTok{\textless{}{-}} \FunctionTok{apply}\NormalTok{(cookies\_data[, }\DecValTok{2}\SpecialCharTok{:}\DecValTok{701}\NormalTok{], }\DecValTok{1}\NormalTok{, max)}


\CommentTok{\# Computation (slope)}

\CommentTok{\# Function: compute\_slope}
\CommentTok{\# @param: spectrum\_values of a cookie (here, column 2 to 701)}
\CommentTok{\# @return: slope of the spectrum curve for a cookie}
\NormalTok{compute\_slope }\OtherTok{\textless{}{-}} \ControlFlowTok{function}\NormalTok{(spectrum\_values) \{}
\NormalTok{  pos }\OtherTok{\textless{}{-}} \DecValTok{1}\SpecialCharTok{:}\FunctionTok{length}\NormalTok{(spectrum\_values)}
\NormalTok{  lm\_model }\OtherTok{\textless{}{-}} \FunctionTok{lm}\NormalTok{(spectrum\_values }\SpecialCharTok{\textasciitilde{}}\NormalTok{ pos)}
\NormalTok{  slope }\OtherTok{\textless{}{-}} \FunctionTok{coef}\NormalTok{(lm\_model)[}\DecValTok{2}\NormalTok{]}
  \FunctionTok{return}\NormalTok{(slope)}
\NormalTok{\}}

\NormalTok{cookies\_data}\SpecialCharTok{$}\NormalTok{slope }\OtherTok{\textless{}{-}} \FunctionTok{apply}\NormalTok{(cookies\_data[, }\DecValTok{2}\SpecialCharTok{:}\DecValTok{701}\NormalTok{], }\DecValTok{1}\NormalTok{, compute\_slope) }


\CommentTok{\# Display of the new columns}
\FunctionTok{head}\NormalTok{(cookies\_data[,}\DecValTok{702}\SpecialCharTok{:}\DecValTok{706}\NormalTok{])}
\end{Highlighting}
\end{Shaded}

\begin{verbatim}
##        mean     stDev      min     max       slope
## 1 0.9851499 0.4111868 0.259270 1.73946 0.001914311
## 2 1.0355417 0.4123933 0.266864 1.66273 0.001898164
## 3 1.0010620 0.4025158 0.251654 1.60960 0.001860203
## 4 1.0280481 0.4040351 0.277777 1.63881 0.001861782
## 5 1.0655011 0.4158252 0.288328 1.70320 0.001910926
## 6 1.0840236 0.4262425 0.284625 1.74356 0.001967228
\end{verbatim}

\begin{center}\rule{0.5\linewidth}{0.5pt}\end{center}

\subsection{Regression model}\label{regression-model}

Now, we have the different features of the spectra.

\begin{Shaded}
\begin{Highlighting}[]
\CommentTok{\# Only features and fat values are retrieved}

\NormalTok{cookies\_features }\OtherTok{\textless{}{-}}\NormalTok{ cookies\_data[}\FunctionTok{c}\NormalTok{(}\StringTok{"fat"}\NormalTok{, }\StringTok{"mean"}\NormalTok{, }\StringTok{"stDev"}\NormalTok{, }\StringTok{"slope"}\NormalTok{, }\StringTok{"min"}\NormalTok{, }\StringTok{"max"}\NormalTok{)]}
\FunctionTok{head}\NormalTok{(cookies\_features)}
\end{Highlighting}
\end{Shaded}

\begin{verbatim}
##     fat      mean     stDev       slope      min     max
## 1 12.57 0.9851499 0.4111868 0.001914311 0.259270 1.73946
## 2 15.13 1.0355417 0.4123933 0.001898164 0.266864 1.66273
## 3 12.63 1.0010620 0.4025158 0.001860203 0.251654 1.60960
## 4 13.85 1.0280481 0.4040351 0.001861782 0.277777 1.63881
## 5 15.25 1.0655011 0.4158252 0.001910926 0.288328 1.70320
## 6 13.66 1.0840236 0.4262425 0.001967228 0.284625 1.74356
\end{verbatim}

\begin{Shaded}
\begin{Highlighting}[]
\NormalTok{X }\OtherTok{\textless{}{-}} \FunctionTok{as.matrix}\NormalTok{(cookies\_features[, }\SpecialCharTok{{-}}\DecValTok{1}\NormalTok{]) }\CommentTok{\# co{-}variables}
\NormalTok{y }\OtherTok{\textless{}{-}}\NormalTok{ cookies\_features}\SpecialCharTok{$}\NormalTok{fat }\CommentTok{\# target variable}
\end{Highlighting}
\end{Shaded}

\begin{verbatim}
## Loading required package: Matrix
\end{verbatim}

\begin{verbatim}
## Loaded glmnet 4.1-8
\end{verbatim}

\subsubsection{Ridge regression}\label{ridge-regression}

We're going to do the ridge regression first, using a cross validation
k-fold to choose the best value for \(\lambda\).

\begin{Shaded}
\begin{Highlighting}[]
\CommentTok{\# Cross validation}
\NormalTok{lambdas\_log }\OtherTok{\textless{}{-}} \DecValTok{10}\SpecialCharTok{\^{}}\FunctionTok{seq}\NormalTok{(}\SpecialCharTok{{-}}\DecValTok{4}\NormalTok{, }\DecValTok{1}\NormalTok{, }\AttributeTok{length.out =} \DecValTok{100}\NormalTok{)}
\NormalTok{cv\_ridge }\OtherTok{\textless{}{-}} \FunctionTok{cv.glmnet}\NormalTok{(X, y, }\AttributeTok{alpha=}\DecValTok{0}\NormalTok{, }\AttributeTok{lambda =}\NormalTok{ (lambdas\_log), }\AttributeTok{standardize =} \ConstantTok{TRUE}\NormalTok{)}

\FunctionTok{plot}\NormalTok{(cv\_ridge)}
\end{Highlighting}
\end{Shaded}

\includegraphics{TP2_MRR_files/figure-latex/unnamed-chunk-5-1.pdf}

\begin{Shaded}
\begin{Highlighting}[]
\NormalTok{best\_lambda }\OtherTok{\textless{}{-}}\NormalTok{ cv\_ridge}\SpecialCharTok{$}\NormalTok{lambda.min }\CommentTok{\# lambda that gives the lowest MSE}
\FunctionTok{print}\NormalTok{(}\FunctionTok{paste}\NormalTok{(}\StringTok{"The best value for lambda is"}\NormalTok{, best\_lambda))}
\end{Highlighting}
\end{Shaded}

\begin{verbatim}
## [1] "The best value for lambda is 0.00367837977182863"
\end{verbatim}

Let's use AIC and BIC criteria to recheck this value.

\begin{Shaded}
\begin{Highlighting}[]
\CommentTok{\# AIC and BIC}
\NormalTok{n }\OtherTok{\textless{}{-}} \FunctionTok{nrow}\NormalTok{(X)}
\NormalTok{p }\OtherTok{\textless{}{-}} \FunctionTok{ncol}\NormalTok{(X)}
\NormalTok{aic }\OtherTok{\textless{}{-}}\NormalTok{ n }\SpecialCharTok{*} \FunctionTok{log}\NormalTok{(cv\_ridge}\SpecialCharTok{$}\NormalTok{cvm) }\SpecialCharTok{+} \DecValTok{2} \SpecialCharTok{*}\NormalTok{ p}
\NormalTok{bic }\OtherTok{\textless{}{-}}\NormalTok{ n }\SpecialCharTok{*} \FunctionTok{log}\NormalTok{(cv\_ridge}\SpecialCharTok{$}\NormalTok{cvm) }\SpecialCharTok{+} \FunctionTok{log}\NormalTok{(n) }\SpecialCharTok{*}\NormalTok{ p}

\FunctionTok{plot}\NormalTok{(}\FunctionTok{log}\NormalTok{(cv\_ridge}\SpecialCharTok{$}\NormalTok{lambda), aic, }\AttributeTok{col =} \StringTok{"red1"}\NormalTok{, }\AttributeTok{type =} \StringTok{"l"}\NormalTok{, }\AttributeTok{xlim =} \FunctionTok{c}\NormalTok{(}\SpecialCharTok{{-}}\DecValTok{8}\NormalTok{, }\SpecialCharTok{{-}}\DecValTok{2}\NormalTok{), }\AttributeTok{ylab =} \StringTok{"Information Criterion"}\NormalTok{)}
\FunctionTok{lines}\NormalTok{(}\FunctionTok{log}\NormalTok{(cv\_ridge}\SpecialCharTok{$}\NormalTok{lambda), bic, }\AttributeTok{col =} \StringTok{"blue1"}\NormalTok{)}
\FunctionTok{legend}\NormalTok{(}\StringTok{"bottomright"}\NormalTok{, }\AttributeTok{lwd =} \DecValTok{1}\NormalTok{, }\AttributeTok{col =} \FunctionTok{c}\NormalTok{(}\StringTok{"red1"}\NormalTok{, }\StringTok{"blue1"}\NormalTok{), }\AttributeTok{legend =} \FunctionTok{c}\NormalTok{(}\StringTok{"AIC"}\NormalTok{, }\StringTok{"BIC"}\NormalTok{))}
\end{Highlighting}
\end{Shaded}

\includegraphics{TP2_MRR_files/figure-latex/unnamed-chunk-6-1.pdf}

\begin{Shaded}
\begin{Highlighting}[]
\NormalTok{best\_lambda\_aic }\OtherTok{\textless{}{-}}\NormalTok{ cv\_ridge}\SpecialCharTok{$}\NormalTok{lambda[}\FunctionTok{which.min}\NormalTok{(aic)] }\CommentTok{\# lambda that gives the lowest AIC}
\NormalTok{best\_lambda\_bic }\OtherTok{\textless{}{-}}\NormalTok{ cv\_ridge}\SpecialCharTok{$}\NormalTok{lambda[}\FunctionTok{which.min}\NormalTok{(bic)] }\CommentTok{\# lambda that gives the lowest BIC}
\end{Highlighting}
\end{Shaded}

Now we can compare the different values for \(\lambda\) we found.

\begin{Shaded}
\begin{Highlighting}[]
\NormalTok{lambda\_values }\OtherTok{\textless{}{-}} \FunctionTok{c}\NormalTok{(best\_lambda, best\_lambda\_aic, best\_lambda\_bic)}
\NormalTok{lambda\_values}
\end{Highlighting}
\end{Shaded}

\begin{verbatim}
## [1] 0.00367838 0.00367838 0.00367838
\end{verbatim}

There are the same. We can also plot the Regularization Path.

\begin{Shaded}
\begin{Highlighting}[]
\FunctionTok{plot}\NormalTok{(cv\_ridge}\SpecialCharTok{$}\NormalTok{glmnet.fit, }\AttributeTok{xvar =} \StringTok{"lambda"}\NormalTok{, }\AttributeTok{ylim =} \FunctionTok{c}\NormalTok{(}\SpecialCharTok{{-}}\DecValTok{100}\NormalTok{, }\DecValTok{100}\NormalTok{))}
\FunctionTok{abline}\NormalTok{(}\AttributeTok{h =} \DecValTok{0}\NormalTok{, }\AttributeTok{col =} \DecValTok{6}\NormalTok{, }\AttributeTok{lty =} \DecValTok{3}\NormalTok{)}
\FunctionTok{abline}\NormalTok{(}\AttributeTok{v =} \FunctionTok{log}\NormalTok{(best\_lambda), }\AttributeTok{col =} \DecValTok{7}\NormalTok{, }\AttributeTok{lty =} \DecValTok{3}\NormalTok{)}
\FunctionTok{legend}\NormalTok{(}\StringTok{"bottomleft"}\NormalTok{, }\AttributeTok{legend =} \FunctionTok{c}\NormalTok{(}\FunctionTok{colnames}\NormalTok{(X), }\StringTok{"Zero"}\NormalTok{, }\StringTok{"Best Lambda"}\NormalTok{), }\AttributeTok{col =} \DecValTok{1}\SpecialCharTok{:}\DecValTok{7}\NormalTok{, }\AttributeTok{lty =} \DecValTok{1}\NormalTok{)}
\end{Highlighting}
\end{Shaded}

\includegraphics{TP2_MRR_files/figure-latex/unnamed-chunk-8-1.pdf}

\begin{center}\rule{0.5\linewidth}{0.5pt}\end{center}

Now we have the best value for \(\lambda\), we do another ridge
regression with this parameter and there is its results :

\begin{verbatim}
## 6 x 1 sparse Matrix of class "dgCMatrix"
##                        s0
## (Intercept) -5.255615e-01
## mean         2.402408e+01
## stDev        7.325394e+01
## slope       -2.116862e+04
## min         -6.558606e+00
## max          9.705276e-01
\end{verbatim}

\begin{Shaded}
\begin{Highlighting}[]
\NormalTok{predictions }\OtherTok{\textless{}{-}} \FunctionTok{predict}\NormalTok{(best\_model\_ridge, }\AttributeTok{newx =}\NormalTok{ X)}

\CommentTok{\# RMSE}
\NormalTok{rmse }\OtherTok{\textless{}{-}} \FunctionTok{sqrt}\NormalTok{(}\FunctionTok{mean}\NormalTok{((predictions }\SpecialCharTok{{-}}\NormalTok{ y)}\SpecialCharTok{\^{}}\DecValTok{2}\NormalTok{)) }
\FunctionTok{print}\NormalTok{(}\FunctionTok{paste}\NormalTok{(}\StringTok{"RMSE ridge model :"}\NormalTok{, rmse))}
\end{Highlighting}
\end{Shaded}

\begin{verbatim}
## [1] "RMSE ridge model : 0.75569802808987"
\end{verbatim}

\begin{Shaded}
\begin{Highlighting}[]
\CommentTok{\# R\^{}2}
\NormalTok{r\_squared }\OtherTok{\textless{}{-}} \DecValTok{1} \SpecialCharTok{{-}} \FunctionTok{sum}\NormalTok{((y }\SpecialCharTok{{-}}\NormalTok{ predictions)}\SpecialCharTok{\^{}}\DecValTok{2}\NormalTok{) }\SpecialCharTok{/} \FunctionTok{sum}\NormalTok{((y }\SpecialCharTok{{-}} \FunctionTok{mean}\NormalTok{(y))}\SpecialCharTok{\^{}}\DecValTok{2}\NormalTok{)}
\FunctionTok{print}\NormalTok{(}\FunctionTok{paste}\NormalTok{(}\StringTok{"R\^{}2 ridge model :"}\NormalTok{, r\_squared))}
\end{Highlighting}
\end{Shaded}

\begin{verbatim}
## [1] "R^2 ridge model : 0.675807868413113"
\end{verbatim}

\begin{center}\rule{0.5\linewidth}{0.5pt}\end{center}

\paragraph{Conclusion of Ridge
regression}\label{conclusion-of-ridge-regression}

We can see that the coefficient of the slope is very important in
absolute value relatively to the others (variables are scaled). It means
that the slope is a very important feature to predict the fat value of a
cookie. Futhermore, the coefficient of the mean, the standard deviation
and the minimum are not null but negligible compared to the slope. It
means that these features are not very important to predict the fat
value of a cookie, but are more important than the max.

\begin{center}\rule{0.5\linewidth}{0.5pt}\end{center}

\subsubsection{Lasso regression}\label{lasso-regression}

\begin{Shaded}
\begin{Highlighting}[]
\NormalTok{cv\_lasso }\OtherTok{\textless{}{-}} \FunctionTok{cv.glmnet}\NormalTok{(X, y, }\AttributeTok{alpha=}\DecValTok{1}\NormalTok{, }\AttributeTok{lambda =}\NormalTok{ (lambdas\_log), }\AttributeTok{standardize =} \ConstantTok{TRUE}\NormalTok{)}

\FunctionTok{plot}\NormalTok{(cv\_lasso)}
\end{Highlighting}
\end{Shaded}

\includegraphics{TP2_MRR_files/figure-latex/unnamed-chunk-11-1.pdf}

\begin{Shaded}
\begin{Highlighting}[]
\NormalTok{best\_lambda\_lasso }\OtherTok{\textless{}{-}}\NormalTok{ cv\_lasso}\SpecialCharTok{$}\NormalTok{lambda.min }\CommentTok{\# lambda that gives the lowest MSE}
\FunctionTok{print}\NormalTok{(}\FunctionTok{paste}\NormalTok{(}\StringTok{"The best value for lambda is"}\NormalTok{, best\_lambda\_lasso))}
\end{Highlighting}
\end{Shaded}

\begin{verbatim}
## [1] "The best value for lambda is 0.00464158883361278"
\end{verbatim}

Let's use AIC and BIC criteria to recheck this value.

\begin{Shaded}
\begin{Highlighting}[]
\CommentTok{\# AIC and BIC}
\NormalTok{n }\OtherTok{\textless{}{-}} \FunctionTok{nrow}\NormalTok{(X)}
\NormalTok{p }\OtherTok{\textless{}{-}} \FunctionTok{ncol}\NormalTok{(X)}
\NormalTok{lasso\_aic }\OtherTok{\textless{}{-}}\NormalTok{ n }\SpecialCharTok{*} \FunctionTok{log}\NormalTok{(cv\_lasso}\SpecialCharTok{$}\NormalTok{cvm) }\SpecialCharTok{+} \DecValTok{2} \SpecialCharTok{*}\NormalTok{ p}
\NormalTok{lasso\_bic }\OtherTok{\textless{}{-}}\NormalTok{ n }\SpecialCharTok{*} \FunctionTok{log}\NormalTok{(cv\_lasso}\SpecialCharTok{$}\NormalTok{cvm) }\SpecialCharTok{+} \FunctionTok{log}\NormalTok{(n) }\SpecialCharTok{*}\NormalTok{ p}

\FunctionTok{plot}\NormalTok{(}\FunctionTok{log}\NormalTok{(cv\_lasso}\SpecialCharTok{$}\NormalTok{lambda), lasso\_aic, }\AttributeTok{col =} \StringTok{"red1"}\NormalTok{, }\AttributeTok{type =} \StringTok{"l"}\NormalTok{, }\AttributeTok{xlim =} \FunctionTok{c}\NormalTok{(}\SpecialCharTok{{-}}\DecValTok{8}\NormalTok{, }\SpecialCharTok{{-}}\DecValTok{2}\NormalTok{), }\AttributeTok{ylab =} \StringTok{"Information Criterion"}\NormalTok{)}
\FunctionTok{lines}\NormalTok{(}\FunctionTok{log}\NormalTok{(cv\_lasso}\SpecialCharTok{$}\NormalTok{lambda), lasso\_bic, }\AttributeTok{col =} \StringTok{"blue1"}\NormalTok{)}
\FunctionTok{legend}\NormalTok{(}\StringTok{"bottomright"}\NormalTok{, }\AttributeTok{lwd =} \DecValTok{1}\NormalTok{, }\AttributeTok{col =} \FunctionTok{c}\NormalTok{(}\StringTok{"red1"}\NormalTok{, }\StringTok{"blue1"}\NormalTok{), }\AttributeTok{legend =} \FunctionTok{c}\NormalTok{(}\StringTok{"AIC"}\NormalTok{, }\StringTok{"BIC"}\NormalTok{))}
\end{Highlighting}
\end{Shaded}

\includegraphics{TP2_MRR_files/figure-latex/unnamed-chunk-12-1.pdf}

\begin{Shaded}
\begin{Highlighting}[]
\NormalTok{best\_lambda\_lasso\_aic }\OtherTok{\textless{}{-}}\NormalTok{ cv\_lasso}\SpecialCharTok{$}\NormalTok{lambda[}\FunctionTok{which.min}\NormalTok{(lasso\_aic)] }\CommentTok{\# lambda that gives the lowest AIC}
\NormalTok{best\_lambda\_lasso\_bic }\OtherTok{\textless{}{-}}\NormalTok{ cv\_lasso}\SpecialCharTok{$}\NormalTok{lambda[}\FunctionTok{which.min}\NormalTok{(lasso\_bic)] }\CommentTok{\# lambda that gives the lowest BIC}
\end{Highlighting}
\end{Shaded}

Now we can compare the different values for \(\lambda\) we found.

\begin{Shaded}
\begin{Highlighting}[]
\NormalTok{lambda\_lasso\_values }\OtherTok{\textless{}{-}} \FunctionTok{c}\NormalTok{(best\_lambda\_lasso, best\_lambda\_lasso\_aic, best\_lambda\_lasso\_bic)}
\NormalTok{lambda\_lasso\_values}
\end{Highlighting}
\end{Shaded}

\begin{verbatim}
## [1] 0.004641589 0.004641589 0.004641589
\end{verbatim}

There are the same. We can also plot the Regularization Path.

\begin{Shaded}
\begin{Highlighting}[]
\FunctionTok{plot}\NormalTok{(cv\_lasso}\SpecialCharTok{$}\NormalTok{glmnet.fit, }\AttributeTok{xvar =} \StringTok{"lambda"}\NormalTok{, }\AttributeTok{ylim =} \FunctionTok{c}\NormalTok{(}\SpecialCharTok{{-}}\DecValTok{100}\NormalTok{, }\DecValTok{100}\NormalTok{))}
\FunctionTok{abline}\NormalTok{(}\AttributeTok{h =} \DecValTok{0}\NormalTok{, }\AttributeTok{col =} \DecValTok{6}\NormalTok{, }\AttributeTok{lty =} \DecValTok{3}\NormalTok{)}
\FunctionTok{abline}\NormalTok{(}\AttributeTok{v =} \FunctionTok{log}\NormalTok{(best\_lambda\_lasso), }\AttributeTok{col =} \DecValTok{7}\NormalTok{, }\AttributeTok{lty =} \DecValTok{3}\NormalTok{)}
\FunctionTok{legend}\NormalTok{(}\StringTok{"bottomleft"}\NormalTok{, }\AttributeTok{legend =} \FunctionTok{c}\NormalTok{(}\FunctionTok{colnames}\NormalTok{(X), }\StringTok{"Zero"}\NormalTok{, }\StringTok{"Best Lambda"}\NormalTok{), }\AttributeTok{col =} \DecValTok{1}\SpecialCharTok{:}\DecValTok{7}\NormalTok{, }\AttributeTok{lty =} \DecValTok{1}\NormalTok{)}
\end{Highlighting}
\end{Shaded}

\includegraphics{TP2_MRR_files/figure-latex/unnamed-chunk-14-1.pdf}

\begin{center}\rule{0.5\linewidth}{0.5pt}\end{center}

Now we have the best value for \(\lambda\), we do another lasso
regression with this parameter and there is its results :

\begin{Shaded}
\begin{Highlighting}[]
\NormalTok{best\_model\_lasso }\OtherTok{\textless{}{-}} \FunctionTok{glmnet}\NormalTok{(X, y, }\AttributeTok{alpha=}\DecValTok{1}\NormalTok{, }\AttributeTok{lambda =}\NormalTok{ best\_lambda\_lasso)}
\FunctionTok{coef}\NormalTok{(best\_model\_lasso)}
\end{Highlighting}
\end{Shaded}

\begin{verbatim}
## 6 x 1 sparse Matrix of class "dgCMatrix"
##                        s0
## (Intercept) -1.091552e+00
## mean         1.908124e+01
## stDev        1.350834e+02
## slope       -3.161498e+04
## min         -1.132358e-01
## max          .
\end{verbatim}

\begin{Shaded}
\begin{Highlighting}[]
\NormalTok{predictions }\OtherTok{\textless{}{-}} \FunctionTok{predict}\NormalTok{(best\_model\_lasso, }\AttributeTok{newx =}\NormalTok{ X)}

\CommentTok{\# RMSE}
\NormalTok{rmse }\OtherTok{\textless{}{-}} \FunctionTok{sqrt}\NormalTok{(}\FunctionTok{mean}\NormalTok{((predictions }\SpecialCharTok{{-}}\NormalTok{ y)}\SpecialCharTok{\^{}}\DecValTok{2}\NormalTok{)) }
\FunctionTok{print}\NormalTok{(}\FunctionTok{paste}\NormalTok{(}\StringTok{"RMSE lasso model :"}\NormalTok{, rmse))}
\end{Highlighting}
\end{Shaded}

\begin{verbatim}
## [1] "RMSE lasso model : 0.73647299949142"
\end{verbatim}

\begin{Shaded}
\begin{Highlighting}[]
\CommentTok{\# R\^{}2}
\NormalTok{r\_squared }\OtherTok{\textless{}{-}} \DecValTok{1} \SpecialCharTok{{-}} \FunctionTok{sum}\NormalTok{((y }\SpecialCharTok{{-}}\NormalTok{ predictions)}\SpecialCharTok{\^{}}\DecValTok{2}\NormalTok{) }\SpecialCharTok{/} \FunctionTok{sum}\NormalTok{((y }\SpecialCharTok{{-}} \FunctionTok{mean}\NormalTok{(y))}\SpecialCharTok{\^{}}\DecValTok{2}\NormalTok{)}
\FunctionTok{print}\NormalTok{(}\FunctionTok{paste}\NormalTok{(}\StringTok{"R\^{}2 lasso model :"}\NormalTok{, r\_squared))}
\end{Highlighting}
\end{Shaded}

\begin{verbatim}
## [1] "R^2 lasso model : 0.692093008165872"
\end{verbatim}

\begin{center}\rule{0.5\linewidth}{0.5pt}\end{center}

\subsubsection{Conclusion of Lasso
regression}\label{conclusion-of-lasso-regression}

We can see that the coefficient of the slope is also very important in
absolute value relatively to the others (variables are scaled). It means
that the slope is a very important feature to predict the fat value of a
cookie. In this case, we can also see that the mean is not null wheras
all the others are. It means that the mean could be a feature to predict
the fat value of a cookie relatively to the others, but is less
important than the slope.

\begin{center}\rule{0.5\linewidth}{0.5pt}\end{center}

\subsection{No penalization}\label{no-penalization}

\begin{Shaded}
\begin{Highlighting}[]
\NormalTok{model\_linear }\OtherTok{\textless{}{-}} \FunctionTok{lm}\NormalTok{(y }\SpecialCharTok{\textasciitilde{}}\NormalTok{ X)}

\NormalTok{predictions\_linear }\OtherTok{\textless{}{-}} \FunctionTok{predict}\NormalTok{(model\_linear, }\AttributeTok{newdata =} \FunctionTok{data.frame}\NormalTok{(X))}

\CommentTok{\# RMSE}
\NormalTok{rmse\_linear }\OtherTok{\textless{}{-}} \FunctionTok{sqrt}\NormalTok{(}\FunctionTok{mean}\NormalTok{((predictions\_linear }\SpecialCharTok{{-}}\NormalTok{ y)}\SpecialCharTok{\^{}}\DecValTok{2}\NormalTok{)) }
\FunctionTok{print}\NormalTok{(}\FunctionTok{paste}\NormalTok{(}\StringTok{"RMSE linear model :"}\NormalTok{, rmse\_linear))}
\end{Highlighting}
\end{Shaded}

\begin{verbatim}
## [1] "RMSE linear model : 0.706440945059218"
\end{verbatim}

\begin{Shaded}
\begin{Highlighting}[]
\CommentTok{\# R\^{}2}
\NormalTok{r\_squared\_linear }\OtherTok{\textless{}{-}} \DecValTok{1} \SpecialCharTok{{-}} \FunctionTok{sum}\NormalTok{((y }\SpecialCharTok{{-}}\NormalTok{ predictions\_linear)}\SpecialCharTok{\^{}}\DecValTok{2}\NormalTok{) }\SpecialCharTok{/} \FunctionTok{sum}\NormalTok{((y }\SpecialCharTok{{-}} \FunctionTok{mean}\NormalTok{(y))}\SpecialCharTok{\^{}}\DecValTok{2}\NormalTok{)}
\FunctionTok{print}\NormalTok{(}\FunctionTok{paste}\NormalTok{(}\StringTok{"R\^{}2 linear model :"}\NormalTok{, r\_squared\_linear))}
\end{Highlighting}
\end{Shaded}

\begin{verbatim}
## [1] "R^2 linear model : 0.716692796197639"
\end{verbatim}

\begin{center}\rule{0.5\linewidth}{0.5pt}\end{center}

\subsection{Conclusion}\label{conclusion}

We can see that for both Lasso and Ridge regressions, the RMSE is higher
and the R\^{}2 is lower than those of the linear model. It means that
the linear model is better than the Lasso and Ridge models to predict
the fat value of a cookie in our study. There could be several reasons
for that. First, the number of features is not very high, so the
penalization is not very useful and we actually lost to much
information. Secondly, the model was not overfitting. Still, the values
of the RMSE and R\^{}2 are very close, so the difference is not very
important, and those penalizations show us that the slope and mean are
indead the most important features to predict the fat value of a cookie.

\end{document}
